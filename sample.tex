\documentclass[uplatex, dvipdfmx]{jsarticle}
\usepackage{85}

\setmyheader{ヘッダー}

\begin{document}

\begin{section}{章}

 テスト。

 \begin{subsection}{節}
     \LaTeX が書けます。
 \end{subsection}

 \begin{figure}[H]
     \centering
     \includegraphics[width=5cm]{sample/sample.jpg}
     % https://www.pexels.com/ja-jp/photo/2071873/
     \caption{これはネコ}
     \label{fig:cat}
 \end{figure}

 \cref{fig:cat}のように引用できます。

 \begin{align}
     a & = b \label{eq:siki1} \\
     b & = c \label{eq:siki2}
 \end{align}

 数式も\cref{eq:siki2}のように引用できます。

 コードもこのように貼り付けられます。

 \mycfile{sample/sample.c}

 SI単位系がいい感じに書けます。$\SI{1.23e-4}{\micro \farad}$のように。

 この表のように揃えることもできます。\footnote{\href{https://uec.medit.link/latex/tablealign.html}{表の小数点揃え|実験AのためのLaTeX小技集}}

 \begin{table}[H]
     \centering
     \caption{サイコロを振った結果($N=600$)}
     \begin{tabular}{cS[table-format=3.0]cS[table-format=+1.1e+1]} \hline
         目 & {回数$n$} & 割合$n/N$ & {理論値との差$(n/N - 1/6)$} \\ \hline
         1 & 88      & 0.147   & -2.0e-2               \\
         2 & 102     & 0.170   & 3e-3                  \\
         3 & 84      & 0.140   & -2.7e-2               \\
         4 & 114     & 0.190   & 2.3e-2                \\
         5 & 109     & 0.182   & 1.5e-2                \\
         6 & 103     & 0.172   & 5e-3                  \\ \hline
     \end{tabular}
 \end{table}

 \sisetup{separate-uncertainty}
 \begin{table}[H]
     \centering
     \caption{不確かさを揃える}
     \begin{tabular}{cS[table-format=2.3,table-figures-uncertainty=1]} \hline
           & {見出し}         \\ \hline
         1 & 0.050+- 0.001 \\
         2 & 0.50 +- 0.01  \\
         3 & 5.0  +- 0.1   \\
         4 & 50    +- 1    \\ \hline
     \end{tabular}
 \end{table}

 \begin{figure}[H]
     \begin{center}
         \begin{circuitikz}[american currents]
             \ctikzset{resistors/scale=0.8, capacitors/scale=0.7}
             \draw
             (0,2) node[anchor=east]{}
             to[normal closed switch=$S_1$, o-] ++(1,0)
             to[short, -] ++(1,0)
             to[generic=$R$] ++(1,0)
             to[short, -*, i^=$i$] ++(1,0)
             to[C=$C$, -*] ++(0,-2)
             to[short, -]++(-1,0)
             to[short, -, i^=$i$] ++(-1,0)
             to[short, -o]++(-2,0)

             (1,2)
             to[normal open switch=$S_2$, *-*] ++(0,-2)

             (4,2)
             to[short, -o] ++(1,0)

             (4,0)
             to[short, -o] ++(1, 0)

             (0,0)
             to[open, v^=$V$]++(0,2)

             (5,0)
             to[open, v=$v_C$]++(0,2)
             ;
         \end{circuitikz}
         \caption{RC直列回路}
         \label{fig:RC1}
     \end{center}
 \end{figure}

 回路も\cref{fig:RC1}のように書けます。めんどくさいです。

\end{section}

\end{document}
